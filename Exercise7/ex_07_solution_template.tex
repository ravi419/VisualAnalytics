\documentclass[11pt,]{article}
\usepackage{lmodern}
\usepackage{amssymb,amsmath}
\usepackage{ifxetex,ifluatex}
\usepackage{fixltx2e} % provides \textsubscript
\ifnum 0\ifxetex 1\fi\ifluatex 1\fi=0 % if pdftex
  \usepackage[T1]{fontenc}
  \usepackage[utf8]{inputenc}
\else % if luatex or xelatex
  \ifxetex
    \usepackage{mathspec}
  \else
    \usepackage{fontspec}
  \fi
  \defaultfontfeatures{Ligatures=TeX,Scale=MatchLowercase}
\fi
% use upquote if available, for straight quotes in verbatim environments
\IfFileExists{upquote.sty}{\usepackage{upquote}}{}
% use microtype if available
\IfFileExists{microtype.sty}{%
\usepackage{microtype}
\UseMicrotypeSet[protrusion]{basicmath} % disable protrusion for tt fonts
}{}
\usepackage[margin=1in]{geometry}
\usepackage{hyperref}
\hypersetup{unicode=true,
            pdftitle={Excersice Sheet 6},
            pdfborder={0 0 0},
            breaklinks=true}
\urlstyle{same}  % don't use monospace font for urls
\usepackage{color}
\usepackage{fancyvrb}
\newcommand{\VerbBar}{|}
\newcommand{\VERB}{\Verb[commandchars=\\\{\}]}
\DefineVerbatimEnvironment{Highlighting}{Verbatim}{commandchars=\\\{\}}
% Add ',fontsize=\small' for more characters per line
\usepackage{framed}
\definecolor{shadecolor}{RGB}{248,248,248}
\newenvironment{Shaded}{\begin{snugshade}}{\end{snugshade}}
\newcommand{\AlertTok}[1]{\textcolor[rgb]{0.94,0.16,0.16}{#1}}
\newcommand{\AnnotationTok}[1]{\textcolor[rgb]{0.56,0.35,0.01}{\textbf{\textit{#1}}}}
\newcommand{\AttributeTok}[1]{\textcolor[rgb]{0.77,0.63,0.00}{#1}}
\newcommand{\BaseNTok}[1]{\textcolor[rgb]{0.00,0.00,0.81}{#1}}
\newcommand{\BuiltInTok}[1]{#1}
\newcommand{\CharTok}[1]{\textcolor[rgb]{0.31,0.60,0.02}{#1}}
\newcommand{\CommentTok}[1]{\textcolor[rgb]{0.56,0.35,0.01}{\textit{#1}}}
\newcommand{\CommentVarTok}[1]{\textcolor[rgb]{0.56,0.35,0.01}{\textbf{\textit{#1}}}}
\newcommand{\ConstantTok}[1]{\textcolor[rgb]{0.00,0.00,0.00}{#1}}
\newcommand{\ControlFlowTok}[1]{\textcolor[rgb]{0.13,0.29,0.53}{\textbf{#1}}}
\newcommand{\DataTypeTok}[1]{\textcolor[rgb]{0.13,0.29,0.53}{#1}}
\newcommand{\DecValTok}[1]{\textcolor[rgb]{0.00,0.00,0.81}{#1}}
\newcommand{\DocumentationTok}[1]{\textcolor[rgb]{0.56,0.35,0.01}{\textbf{\textit{#1}}}}
\newcommand{\ErrorTok}[1]{\textcolor[rgb]{0.64,0.00,0.00}{\textbf{#1}}}
\newcommand{\ExtensionTok}[1]{#1}
\newcommand{\FloatTok}[1]{\textcolor[rgb]{0.00,0.00,0.81}{#1}}
\newcommand{\FunctionTok}[1]{\textcolor[rgb]{0.00,0.00,0.00}{#1}}
\newcommand{\ImportTok}[1]{#1}
\newcommand{\InformationTok}[1]{\textcolor[rgb]{0.56,0.35,0.01}{\textbf{\textit{#1}}}}
\newcommand{\KeywordTok}[1]{\textcolor[rgb]{0.13,0.29,0.53}{\textbf{#1}}}
\newcommand{\NormalTok}[1]{#1}
\newcommand{\OperatorTok}[1]{\textcolor[rgb]{0.81,0.36,0.00}{\textbf{#1}}}
\newcommand{\OtherTok}[1]{\textcolor[rgb]{0.56,0.35,0.01}{#1}}
\newcommand{\PreprocessorTok}[1]{\textcolor[rgb]{0.56,0.35,0.01}{\textit{#1}}}
\newcommand{\RegionMarkerTok}[1]{#1}
\newcommand{\SpecialCharTok}[1]{\textcolor[rgb]{0.00,0.00,0.00}{#1}}
\newcommand{\SpecialStringTok}[1]{\textcolor[rgb]{0.31,0.60,0.02}{#1}}
\newcommand{\StringTok}[1]{\textcolor[rgb]{0.31,0.60,0.02}{#1}}
\newcommand{\VariableTok}[1]{\textcolor[rgb]{0.00,0.00,0.00}{#1}}
\newcommand{\VerbatimStringTok}[1]{\textcolor[rgb]{0.31,0.60,0.02}{#1}}
\newcommand{\WarningTok}[1]{\textcolor[rgb]{0.56,0.35,0.01}{\textbf{\textit{#1}}}}
\usepackage{longtable,booktabs}
\usepackage{graphicx,grffile}
\makeatletter
\def\maxwidth{\ifdim\Gin@nat@width>\linewidth\linewidth\else\Gin@nat@width\fi}
\def\maxheight{\ifdim\Gin@nat@height>\textheight\textheight\else\Gin@nat@height\fi}
\makeatother
% Scale images if necessary, so that they will not overflow the page
% margins by default, and it is still possible to overwrite the defaults
% using explicit options in \includegraphics[width, height, ...]{}
\setkeys{Gin}{width=\maxwidth,height=\maxheight,keepaspectratio}
\IfFileExists{parskip.sty}{%
\usepackage{parskip}
}{% else
\setlength{\parindent}{0pt}
\setlength{\parskip}{6pt plus 2pt minus 1pt}
}
\setlength{\emergencystretch}{3em}  % prevent overfull lines
\providecommand{\tightlist}{%
  \setlength{\itemsep}{0pt}\setlength{\parskip}{0pt}}
\setcounter{secnumdepth}{0}
% Redefines (sub)paragraphs to behave more like sections
\ifx\paragraph\undefined\else
\let\oldparagraph\paragraph
\renewcommand{\paragraph}[1]{\oldparagraph{#1}\mbox{}}
\fi
\ifx\subparagraph\undefined\else
\let\oldsubparagraph\subparagraph
\renewcommand{\subparagraph}[1]{\oldsubparagraph{#1}\mbox{}}
\fi

%%% Use protect on footnotes to avoid problems with footnotes in titles
\let\rmarkdownfootnote\footnote%
\def\footnote{\protect\rmarkdownfootnote}

%%% Change title format to be more compact
\usepackage{titling}

% Create subtitle command for use in maketitle
\providecommand{\subtitle}[1]{
  \posttitle{
    \begin{center}\large#1\end{center}
    }
}

\setlength{\droptitle}{-2em}

  \title{Excersice Sheet 6}
    \pretitle{\vspace{\droptitle}\centering\huge}
  \posttitle{\par}
    \author{}
    \preauthor{}\postauthor{}
    \date{}
    \predate{}\postdate{}
  
\usepackage[german]{babel}
\usepackage{caption}

\begin{document}
\maketitle

\captionsetup[table]{labelformat=empty}

A broker wants to use linear regression to find out which factors have a
large influence on the price of a property. For this purpose, the
variables described in Table 1 are given for the last 88 sales in the
broker's region.

\begin{longtable}[]{@{}ll@{}}
\caption{Table 1 House price record}\tabularnewline
\toprule
Variabel & Description\tabularnewline
\midrule
\endfirsthead
\toprule
Variabel & Description\tabularnewline
\midrule
\endhead
\texttt{price} & house price (\(\times\) 1,000 EUR)\tabularnewline
\texttt{bdrms} & number bedrooms\tabularnewline
\texttt{lotsize} & parking area (m\(^2\))\tabularnewline
\texttt{sqrm} & house area (m\(^2\))\tabularnewline
\texttt{country} & \(==1\) when in country house style\tabularnewline
\texttt{lprice} & \texttt{log(price)}\tabularnewline
\texttt{llotsize} & \texttt{log(lotsize)}\tabularnewline
\texttt{lsqrm} & \texttt{log(sqrm)}\tabularnewline
\bottomrule
\end{longtable}

\begin{enumerate}
\def\labelenumi{\arabic{enumi}.}
\tightlist
\item
  Create a linear regression model with \texttt{price} as dependent
  variable and \texttt{bdrms}, \texttt{lotsize}, \texttt{sqrm} und
  \texttt{country} as independent variables.

  \begin{enumerate}
  \def\labelenumii{\alph{enumii})}
  \tightlist
  \item
    Determine the regression coefficients and \(p\)-values of the
    dependent variable and compare their influence within the model on
    the predicted value for \texttt{price}.
  \item
    Determine how much variance of the dependent variable is explained.
  \item
    Check the residuals (graphically) for normal distribution and
    homoskedasticity.
  \end{enumerate}
\end{enumerate}

\begin{Shaded}
\begin{Highlighting}[]
\CommentTok{# Solution for Task 1...}
\KeywordTok{library}\NormalTok{(stringr)}
\KeywordTok{library}\NormalTok{(readr)}
\KeywordTok{library}\NormalTok{(dplyr)}
\KeywordTok{library}\NormalTok{(psych)}
\KeywordTok{library}\NormalTok{(caTools)}
\KeywordTok{library}\NormalTok{(PerformanceAnalytics)}

\NormalTok{dataset <-}\StringTok{ }\KeywordTok{read_csv}\NormalTok{(}\StringTok{"hprice.csv"}\NormalTok{)}

\NormalTok{dataset.reg <-}\StringTok{ }\KeywordTok{subset}\NormalTok{(dataset, }\DataTypeTok{select=}\KeywordTok{c}\NormalTok{(price,bdrms, lotsize, sqrm,country))}

\KeywordTok{pairs.panels}\NormalTok{(dataset.reg,}\DataTypeTok{col=}\StringTok{"red"}\NormalTok{)}
\end{Highlighting}
\end{Shaded}

\includegraphics{ex_07_solution_template_files/figure-latex/unnamed-chunk-1-1.pdf}

\begin{Shaded}
\begin{Highlighting}[]
\KeywordTok{chart.Correlation}\NormalTok{(dataset.reg,}\DataTypeTok{hist=}\NormalTok{T)}
\end{Highlighting}
\end{Shaded}

\includegraphics{ex_07_solution_template_files/figure-latex/unnamed-chunk-1-2.pdf}

\begin{Shaded}
\begin{Highlighting}[]
\NormalTok{regressor =}\StringTok{ }\KeywordTok{lm}\NormalTok{(}\DataTypeTok{formula =}\NormalTok{ price }\OperatorTok{~}\StringTok{ }\NormalTok{.,}
               \DataTypeTok{data =}\NormalTok{ dataset.reg)}

\KeywordTok{print}\NormalTok{(}\KeywordTok{summary}\NormalTok{(regressor)}\OperatorTok{$}\NormalTok{coefficient)}
\end{Highlighting}
\end{Shaded}

\begin{verbatim}
##                 Estimate   Std. Error    t value     Pr(>|t|)
## (Intercept) -24.12652827 29.603454535 -0.8149903 4.174103e-01
## bdrms        11.00429220  9.515260399  1.1564888 2.507991e-01
## lotsize       0.02234481  0.006917665  3.2301082 1.774189e-03
## sqrm          1.33732481  0.143576518  9.3143700 1.534380e-14
## country      13.71554214 14.637265211  0.9370290 3.514622e-01
\end{verbatim}

\begin{Shaded}
\begin{Highlighting}[]
\KeywordTok{hist}\NormalTok{(}\KeywordTok{residuals}\NormalTok{(regressor), }\CommentTok{# histogram}
     \DataTypeTok{col=}\StringTok{"cyan"}\NormalTok{, }\CommentTok{# column color}
     \DataTypeTok{border=}\StringTok{"red"}\NormalTok{,}
     \DataTypeTok{prob =} \OtherTok{TRUE}\NormalTok{, }\CommentTok{# show densities instead of frequencies}
     \DataTypeTok{xlab =} \StringTok{"range"}\NormalTok{,}
     \DataTypeTok{main =} \StringTok{"Reseduals"}\NormalTok{)}
\KeywordTok{lines}\NormalTok{(}\KeywordTok{density}\NormalTok{(}\KeywordTok{residuals}\NormalTok{(regressor)), }\CommentTok{# density plot}
      \DataTypeTok{lwd =} \DecValTok{2}\NormalTok{, }\CommentTok{# thickness of line}
      \DataTypeTok{col =} \StringTok{"black"}\NormalTok{)}
\end{Highlighting}
\end{Shaded}

\includegraphics{ex_07_solution_template_files/figure-latex/unnamed-chunk-1-3.pdf}

\begin{enumerate}
\def\labelenumi{\arabic{enumi}.}
\setcounter{enumi}{1}
\tightlist
\item
  Given be the linear regression model from task 1.

  \begin{enumerate}
  \def\labelenumii{\alph{enumii})}
  \tightlist
  \item
    Create a scatterplot to display the relationship between the
    predicted value for \texttt{price} and the residual size.
  \item
    For some houses, the price forecast of the broker model is more than
    EUR 100,000 off. Highlight houses with a residual size of more than
    100 or less than 100. What could be the reasons for high model
    inaccuracies?
  \item
    Can the \(R^2\)-value be increased by using a linear transformation
    of one of the independent variables?
  \end{enumerate}
\end{enumerate}

\begin{Shaded}
\begin{Highlighting}[]
\CommentTok{# Solution for Task 2...}
\KeywordTok{par}\NormalTok{(}\DataTypeTok{mfrow=}\KeywordTok{c}\NormalTok{(}\DecValTok{2}\NormalTok{,}\DecValTok{2}\NormalTok{))}
\KeywordTok{plot}\NormalTok{(regressor)}
\end{Highlighting}
\end{Shaded}

\includegraphics{ex_07_solution_template_files/figure-latex/unnamed-chunk-2-1.pdf}

\begin{enumerate}
\def\labelenumi{\arabic{enumi}.}
\setcounter{enumi}{2}
\tightlist
\item
  Graphically display the relationship between \texttt{bdrms} and
  \texttt{price}. Check whether this relationship is also reflected in
  the regression model from Task 1. Create a regression model with
  \texttt{bdrms} as the only independent variable. Compare the
  regression coefficients with those of the model from Task 1 and
  interpret the differences.
\end{enumerate}

\begin{Shaded}
\begin{Highlighting}[]
\CommentTok{# Solution for Task 3...}

\NormalTok{dataset.bedrom <-}\StringTok{ }\KeywordTok{subset}\NormalTok{(dataset, }\DataTypeTok{select=}\KeywordTok{c}\NormalTok{(price,bdrms))}

\KeywordTok{chart.Correlation}\NormalTok{(dataset.bedrom,}\DataTypeTok{hist=}\NormalTok{T)}
\end{Highlighting}
\end{Shaded}

\includegraphics{ex_07_solution_template_files/figure-latex/unnamed-chunk-3-1.pdf}

\begin{Shaded}
\begin{Highlighting}[]
\NormalTok{model <-}\StringTok{ }\KeywordTok{lm}\NormalTok{(}\DataTypeTok{fromula=}\NormalTok{ price }\OperatorTok{~}\StringTok{ }\NormalTok{bedrms,}\DataTypeTok{data =}\NormalTok{ dataset.bedrom)}
\end{Highlighting}
\end{Shaded}

\begin{center}\rule{0.5\linewidth}{\linethickness}\end{center}

Dataset:

\begin{itemize}
\tightlist
\item
  \url{http://isgwww.cs.uni-magdeburg.de/cv/lehre/VisAnalytics/material/exercise/datasets/hprice.csv}
\end{itemize}


\end{document}
