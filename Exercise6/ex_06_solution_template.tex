\PassOptionsToPackage{unicode=true}{hyperref} % options for packages loaded elsewhere
\PassOptionsToPackage{hyphens}{url}
%
\documentclass[11pt,]{article}
\usepackage{lmodern}
\usepackage{amssymb,amsmath}
\usepackage{ifxetex,ifluatex}
\usepackage{fixltx2e} % provides \textsubscript
\ifnum 0\ifxetex 1\fi\ifluatex 1\fi=0 % if pdftex
  \usepackage[T1]{fontenc}
  \usepackage[utf8]{inputenc}
  \usepackage{textcomp} % provides euro and other symbols
\else % if luatex or xelatex
  \usepackage{unicode-math}
  \defaultfontfeatures{Ligatures=TeX,Scale=MatchLowercase}
\fi
% use upquote if available, for straight quotes in verbatim environments
\IfFileExists{upquote.sty}{\usepackage{upquote}}{}
% use microtype if available
\IfFileExists{microtype.sty}{%
\usepackage[]{microtype}
\UseMicrotypeSet[protrusion]{basicmath} % disable protrusion for tt fonts
}{}
\IfFileExists{parskip.sty}{%
\usepackage{parskip}
}{% else
\setlength{\parindent}{0pt}
\setlength{\parskip}{6pt plus 2pt minus 1pt}
}
\usepackage{hyperref}
\hypersetup{
            pdftitle={Excersice Sheet 6},
            pdfborder={0 0 0},
            breaklinks=true}
\urlstyle{same}  % don't use monospace font for urls
\usepackage[margin=1in]{geometry}
\usepackage{color}
\usepackage{fancyvrb}
\newcommand{\VerbBar}{|}
\newcommand{\VERB}{\Verb[commandchars=\\\{\}]}
\DefineVerbatimEnvironment{Highlighting}{Verbatim}{commandchars=\\\{\}}
% Add ',fontsize=\small' for more characters per line
\newenvironment{Shaded}{}{}
\newcommand{\AlertTok}[1]{\textcolor[rgb]{1.00,0.00,0.00}{#1}}
\newcommand{\AnnotationTok}[1]{\textcolor[rgb]{0.00,0.50,0.00}{#1}}
\newcommand{\AttributeTok}[1]{#1}
\newcommand{\BaseNTok}[1]{#1}
\newcommand{\BuiltInTok}[1]{#1}
\newcommand{\CharTok}[1]{\textcolor[rgb]{0.00,0.50,0.50}{#1}}
\newcommand{\CommentTok}[1]{\textcolor[rgb]{0.00,0.50,0.00}{#1}}
\newcommand{\CommentVarTok}[1]{\textcolor[rgb]{0.00,0.50,0.00}{#1}}
\newcommand{\ConstantTok}[1]{#1}
\newcommand{\ControlFlowTok}[1]{\textcolor[rgb]{0.00,0.00,1.00}{#1}}
\newcommand{\DataTypeTok}[1]{#1}
\newcommand{\DecValTok}[1]{#1}
\newcommand{\DocumentationTok}[1]{\textcolor[rgb]{0.00,0.50,0.00}{#1}}
\newcommand{\ErrorTok}[1]{\textcolor[rgb]{1.00,0.00,0.00}{\textbf{#1}}}
\newcommand{\ExtensionTok}[1]{#1}
\newcommand{\FloatTok}[1]{#1}
\newcommand{\FunctionTok}[1]{#1}
\newcommand{\ImportTok}[1]{#1}
\newcommand{\InformationTok}[1]{\textcolor[rgb]{0.00,0.50,0.00}{#1}}
\newcommand{\KeywordTok}[1]{\textcolor[rgb]{0.00,0.00,1.00}{#1}}
\newcommand{\NormalTok}[1]{#1}
\newcommand{\OperatorTok}[1]{#1}
\newcommand{\OtherTok}[1]{\textcolor[rgb]{1.00,0.25,0.00}{#1}}
\newcommand{\PreprocessorTok}[1]{\textcolor[rgb]{1.00,0.25,0.00}{#1}}
\newcommand{\RegionMarkerTok}[1]{#1}
\newcommand{\SpecialCharTok}[1]{\textcolor[rgb]{0.00,0.50,0.50}{#1}}
\newcommand{\SpecialStringTok}[1]{\textcolor[rgb]{0.00,0.50,0.50}{#1}}
\newcommand{\StringTok}[1]{\textcolor[rgb]{0.00,0.50,0.50}{#1}}
\newcommand{\VariableTok}[1]{#1}
\newcommand{\VerbatimStringTok}[1]{\textcolor[rgb]{0.00,0.50,0.50}{#1}}
\newcommand{\WarningTok}[1]{\textcolor[rgb]{0.00,0.50,0.00}{\textbf{#1}}}
\usepackage{graphicx,grffile}
\makeatletter
\def\maxwidth{\ifdim\Gin@nat@width>\linewidth\linewidth\else\Gin@nat@width\fi}
\def\maxheight{\ifdim\Gin@nat@height>\textheight\textheight\else\Gin@nat@height\fi}
\makeatother
% Scale images if necessary, so that they will not overflow the page
% margins by default, and it is still possible to overwrite the defaults
% using explicit options in \includegraphics[width, height, ...]{}
\setkeys{Gin}{width=\maxwidth,height=\maxheight,keepaspectratio}
\setlength{\emergencystretch}{3em}  % prevent overfull lines
\providecommand{\tightlist}{%
  \setlength{\itemsep}{0pt}\setlength{\parskip}{0pt}}
\setcounter{secnumdepth}{0}
% Redefines (sub)paragraphs to behave more like sections
\ifx\paragraph\undefined\else
\let\oldparagraph\paragraph
\renewcommand{\paragraph}[1]{\oldparagraph{#1}\mbox{}}
\fi
\ifx\subparagraph\undefined\else
\let\oldsubparagraph\subparagraph
\renewcommand{\subparagraph}[1]{\oldsubparagraph{#1}\mbox{}}
\fi

% set default figure placement to htbp
\makeatletter
\def\fps@figure{htbp}
\makeatother

\usepackage[german]{babel}

\title{Excersice Sheet 6}
\author{}
\date{\vspace{-2.5em}}

\begin{document}
\maketitle

\emph{What factors explain excessive alcohol consumption among
students?} The record for the task sheet comes from a survey of students
who attended mathematics and Portuguese courses and contains many
interesting details about their sociodemographics, life circumstances
and learning success.\\
The ordinal scaled variables \texttt{Dalc} and \texttt{Walc} give
information about the alcohol consumption of the students on weekdays
and weekends. Create a binary target variable \texttt{alc\_prob} as
follows:

\begin{Shaded}
\begin{Highlighting}[]
\KeywordTok{library}\NormalTok{(stringr)}
\KeywordTok{library}\NormalTok{(readr)}
\KeywordTok{library}\NormalTok{(dplyr)}
\CommentTok{# (Adapt Path)}
\NormalTok{ StudentDF <-}\StringTok{ }\KeywordTok{read_csv}\NormalTok{(}\KeywordTok{str_c}\NormalTok{(}\KeywordTok{dirname}\NormalTok{(}\KeywordTok{getwd}\NormalTok{()), }\StringTok{"/Exercise6/student_alc.csv"}\NormalTok{))}
 
\NormalTok{ gain.function <-}\StringTok{ }\ControlFlowTok{function}\NormalTok{(GiniTar,s_var)\{}
\NormalTok{  g =}\StringTok{ }\NormalTok{GiniTar }\OperatorTok{-}\StringTok{ }\NormalTok{s_var}
  \KeywordTok{return}\NormalTok{(g)}
\NormalTok{\}}
\NormalTok{ Ginni.function <-}\StringTok{ }\ControlFlowTok{function}\NormalTok{(DF)\{}
\NormalTok{  p <<-}\StringTok{ }\NormalTok{(}\KeywordTok{apply}\NormalTok{(DF, }\DecValTok{2}\NormalTok{, }\ControlFlowTok{function}\NormalTok{(i) i}\OperatorTok{/}\KeywordTok{sum}\NormalTok{(i)))}
\NormalTok{  Gin <-}\StringTok{ }\DecValTok{1}\OperatorTok{-}\StringTok{ }\NormalTok{(}\KeywordTok{sum}\NormalTok{(p}\OperatorTok{*}\NormalTok{p))}

  \KeywordTok{return}\NormalTok{(Gin)}
\NormalTok{\}}
\NormalTok{main.function<-}\StringTok{ }\ControlFlowTok{function}\NormalTok{(df)\{}
\NormalTok{  StudentDF <<-}\StringTok{ }\NormalTok{df }\OperatorTok\StringTok{ }\KeywordTok{mutate}\NormalTok{(}\DataTypeTok{alc_prob =} \KeywordTok{ifelse}\NormalTok{(Dalc }\OperatorTok{+}\StringTok{ }\NormalTok{Walc }\OperatorTok{>=}\StringTok{ }\DecValTok{6}\NormalTok{, }\StringTok{"alc_p"}\NormalTok{ , }\StringTok{"no_alc_p"}\NormalTok{ ))}
  \KeywordTok{colnames}\NormalTok{(StudentDF)[}\KeywordTok{ncol}\NormalTok{(StudentDF)] <-}\StringTok{ "TargVar"}
\NormalTok{  TargetList <-}\StringTok{ }\KeywordTok{as.data.frame}\NormalTok{(}\KeywordTok{table}\NormalTok{(StudentDF}\OperatorTok{$}\NormalTok{TargVar))}
\NormalTok{  TargetList <-}\StringTok{ }\KeywordTok{data.frame}\NormalTok{(TargetList[,}\OperatorTok{-}\DecValTok{1}\NormalTok{], }\DataTypeTok{row.names =}\NormalTok{ TargetList[,}\DecValTok{1}\NormalTok{])}
\NormalTok{  GiniTar <-}\StringTok{ }\KeywordTok{Ginni.function}\NormalTok{(TargetList)}
\NormalTok{  Student_alc_prob =}\StringTok{ }\KeywordTok{subset}\NormalTok{(StudentDF, }\DataTypeTok{select=} \OperatorTok{-}\KeywordTok{c}\NormalTok{(TargVar))}
\NormalTok{  GiniGainList =}\StringTok{ }\KeywordTok{list}\NormalTok{()}
\NormalTok{  gini_var <-}\StringTok{ }\DecValTok{0}
  \ControlFlowTok{for}\NormalTok{ (Var }\ControlFlowTok{in}\NormalTok{ (}\KeywordTok{names}\NormalTok{(Student_alc_prob))) \{}
\NormalTok{    var_df <-}\KeywordTok{count}\NormalTok{(StudentDF,StudentDF[[Var]],StudentDF}\OperatorTok{$}\NormalTok{TargVar)}
    \ControlFlowTok{for}\NormalTok{ (var_val }\ControlFlowTok{in} \KeywordTok{unique}\NormalTok{(var_df[[}\DecValTok{1}\NormalTok{]])) \{}
\NormalTok{      temp_df <-}\StringTok{ }\KeywordTok{subset}\NormalTok{(var_df,var_df[[}\DecValTok{1}\NormalTok{]]}\OperatorTok{==}\NormalTok{var_val)}
\NormalTok{      a<-}\StringTok{ }\KeywordTok{dim}\NormalTok{(StudentDF)}
\NormalTok{      Probablity <-}\StringTok{ }\KeywordTok{sum}\NormalTok{(temp_df}\OperatorTok{$}\NormalTok{n)}\OperatorTok{/}\NormalTok{a[}\DecValTok{1}\NormalTok{]}
\NormalTok{      filtered_df =}\StringTok{ }\KeywordTok{subset}\NormalTok{(temp_df, }\DataTypeTok{select=} \OperatorTok{-}\KeywordTok{c}\NormalTok{(}\DecValTok{1}\OperatorTok{:}\DecValTok{2}\NormalTok{))}
\NormalTok{      gini_val <-}\StringTok{ }\KeywordTok{Ginni.function}\NormalTok{(filtered_df)}\OperatorTok{*}\NormalTok{Probablity}
\NormalTok{      gini_var <-}\StringTok{ }\NormalTok{gini_var}\OperatorTok{+}\StringTok{ }\NormalTok{gini_val}
\NormalTok{    \}}
\NormalTok{    gini_gain <-}\StringTok{ }\KeywordTok{gain.function}\NormalTok{(GiniTar,gini_var)}
\NormalTok{    GiniGainList[[Var]] <-}\StringTok{ }\NormalTok{gini_gain}
\NormalTok{    gini_var <-}\StringTok{ }\DecValTok{0}
\NormalTok{    gini_val<-}\StringTok{ }\DecValTok{0}
\NormalTok{    gini_gain<-}\DecValTok{0}
    
\NormalTok{  \}}
  
\NormalTok{  TotalGinni<-}\StringTok{ }\KeywordTok{do.call}\NormalTok{(rbind, GiniGainList)}
\NormalTok{\}}

\KeywordTok{main.function}\NormalTok{(StudentDF)}
\end{Highlighting}
\end{Shaded}

\begin{enumerate}
\def\labelenumi{\arabic{enumi}.}
\tightlist
\item
  Calculate the Gini index for the target variable \texttt{alc\_prob}
  and the \emph{Gini index} for each variable with respect to
  \texttt{alc\_prob}. Determine the 5 variables with the highest
  \emph{Gini Gain}.
\end{enumerate}

\begin{Shaded}
\begin{Highlighting}[]
\CommentTok{# Solution for Task 1}
\end{Highlighting}
\end{Shaded}

\begin{enumerate}
\def\labelenumi{\arabic{enumi}.}
\setcounter{enumi}{1}
\tightlist
\item
  Learn 2 different decision trees with \texttt{alc\_prob} as target
  variable. For the first tree, nodes should be further partitioned
  until the class distribution of all resulting leaf nodes is pure. For
  the second tree, nodes with a cardinality of less than 20 instances
  should not be further partitioned. Determine the quality of the trees
  by calculating sensitivity (\emph{True Positive Rate}) and specificity
  (\emph{True Negative Rate}) for a 70\%:30\% split in training and test
  sets. Display the decision trees graphically and discuss the
  differences in quality measures
\end{enumerate}

\begin{Shaded}
\begin{Highlighting}[]
\CommentTok{# Solution for Task 2}
\end{Highlighting}
\end{Shaded}

\begin{enumerate}
\def\labelenumi{\arabic{enumi}.}
\setcounter{enumi}{2}
\tightlist
\item
  Use \texttt{randomForest::randomForest()} to create a random forest
  with 200 trees. As candidates for a split within a tree a random
  sample of 5 variables should be drawn. Calculate Accuracy, Sensitivity
  and Specificity for the Out-of-the-Bag instances and show the most
  important variables (\texttt{?importance}).
\end{enumerate}

\begin{Shaded}
\begin{Highlighting}[]
\CommentTok{# Solution for Task 3}
\end{Highlighting}
\end{Shaded}

\begin{center}\rule{0.5\linewidth}{0.5pt}\end{center}

Dataset:
\url{http://isgwww.cs.uni-magdeburg.de/cv/lehre/VisAnalytics/material/exercise/datasets/student_alc.csv}\\
(Source: \url{https://www.kaggle.com/uciml/student-alcohol-consumption})

\end{document}
